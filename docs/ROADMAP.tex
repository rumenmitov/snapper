% Created 2025-06-09 Mon 10:52
% Intended LaTeX compiler: pdflatex
\documentclass[11pt]{article}
\usepackage[utf8]{inputenc}
\usepackage[T1]{fontenc}
\usepackage{graphicx}
\usepackage{longtable}
\usepackage{wrapfig}
\usepackage{rotating}
\usepackage[normalem]{ulem}
\usepackage{amsmath}
\usepackage{amssymb}
\usepackage{capt-of}
\usepackage{hyperref}
\author{Rumen Mitov}
\date{\today}
\title{Snapper 2.0 - Roadmap}
\hypersetup{
 pdfauthor={Rumen Mitov},
 pdftitle={Snapper 2.0 - Roadmap},
 pdfkeywords={},
 pdfsubject={},
 pdfcreator={Emacs 30.1 (Org mode 9.7.11)}, 
 pdflang={English}}
\begin{document}

\maketitle
\section{{\bfseries\sffamily TODO} \framebox{\#A} Setup Snapper Project}
\label{sec:org02254b2}
\noindent\textbf{DEADLINE:} \textit{<2025-06-16 Mon>}\\
\begin{verbatim}
Effort: 4
\end{verbatim}
\begin{itemize}
\item[{$\boxtimes$}] Create run file and Makefile
\item[{$\square$}] \sout{Use Goa as a build tool}
\item[{$\square$}] Define data structures and \texttt{Snapper} object.
\end{itemize}
\section{{\bfseries\sffamily TODO} \framebox{\#C} Unit Tests}
\label{sec:orga2cb21b}
\noindent\textbf{DEADLINE:} \textit{<2025-06-16 Mon>}\\
\begin{verbatim}
Effort: 4
\end{verbatim}
The following unit tests simulate the virtual address space via a linked-list.
\subsection{{\bfseries\sffamily TODO} Snapshot creation}
\label{sec:org138ba3c}
\begin{enumerate}
\item Create a linked list of 1000 elements containing random integers.
\item Create a snapshot of the linked list.
\item If test is successful, there should be 1000 files in the \uline{snapshot} directory of the generation.
\end{enumerate}
\subsection{{\bfseries\sffamily TODO} Snapshot successful recovery}
\label{sec:org0a97a78}
\subsubsection{Test \#1}
\label{sec:org0a721d8}
Note, this test requires a compressed archive of a generation storing integers in the range from 1 to 1000 in increasing order. The archive should be uncompressed before the test is ran, and the uncompressed files should be deleted after the test finishes.

\begin{enumerate}
\item Create an empty linked-list.
\item Recover each file from the generation into an element in the linked-list.
\item If test is successful, the linked-list should store all numbers from 1 to 1000 in ascending order.
\end{enumerate}
\subsubsection{Test \#2}
\label{sec:org6a8744d}
Note, this test requires a compressed archive containing two generations: one invalid, the second invalid. Both generations are snapshots of a linked-list containing all integers from 1 to 1000 in increasing order. However, the older, invalid one contains some files that are needed by the valid generation. 

\begin{enumerate}
\item Create an empty linked-list.
\item Recover each file from the valid generation into an element in the linked-list.
\item If test is successful, the linked-list should store all numbers from 1 to 1000 in ascending order.
\end{enumerate}
\subsubsection{Test \#3}
\label{sec:orgae226a1}
Note, this test requires a compressed archive containing a generation and an incomplete snapshot named "current". The generation should store integers in the range from 1 to 1000 in increasing order.
\begin{enumerate}
\item Create an empty linked-list.
\item Recover each file from the generation into an element in the linked-list.
\item If test is successful, the linked-list should store all numbers from 1 to 1000 in ascending order and the "current" directory should have been removed.
\end{enumerate}
\subsection{{\bfseries\sffamily TODO} Snapshot unsuccessful recovery}
\label{sec:orgb130d74}
Note, this test requires a compressed archive of a generation whose archive file has an invalid CRC. 

\begin{enumerate}
\item Try to recover the generation.
\item If test is successful, the recovery should not be possible and an error should be written stating that the archive file is invalid.
\end{enumerate}
\subsection{{\bfseries\sffamily TODO} Snapshot purge}
\label{sec:orgce3a4a9}
Note, this test requires a compressed archive of two valid generations. Both generations are snapshots of a linked-list containing all integers from 1 to 1000 in increasing order. However, the older one contains some files that are needed by the valid generation. 

\begin{enumerate}
\item Purge the older generation.
\item Create an empty linked-list.
\item Trying to recover from the older generation should be unsuccessful.
\item Recover each file from the non-purged generation into an element in the linked-list.
\item If test is successful, the linked-list should store all numbers from 1 to 1000 in ascending order.
\end{enumerate}
\section{{\bfseries\sffamily TODO} \framebox{\#B} Snapshot Creation}
\label{sec:org306c947}
\noindent\textbf{DEADLINE:} \textit{<2025-06-16 Mon>}\\
\begin{verbatim}
Effort: 10
\end{verbatim}
\begin{enumerate}
\item Delete any unfinished snapshot generation called "current".
\item Initialize a new generation directory called "current".
\item Within the generation directory create the archive file and the snapshot directory.
\item Check if there is a valid prior generation (based on the timestamps). If there is, load the archive file's data into the \texttt{Snapper::Archiver} array.
\item Let h\textsubscript{i} := \texttt{Snapper::Archiver[i]}. If \texttt{Snapper::Archiver[i]} contains redundant files, use the \emph{first file} before the comma.
\item For each virtual page where the CRC of the file h\textsubscript{i} does not match the CRC of page (or h\textsubscript{i} does not exist):
\begin{enumerate}
\item Create new file, h\textsubscript{j}, and save the binary contents of the page into this new file.
\item Initialize the snapshot file with the new CRC of the data, a reference count of 1, and the binary data of the page.
\item Update \texttt{Snapper::Archiver[i]} \(\gets\) \emph{path(} h\textsubscript{j} \emph{)}, there \emph{path()} is the path relative to \uline{<snapper-root>}.
\end{enumerate}
\item For each virtual page where CRC of the file h\textsubscript{i} matches the CRC of the page:
\begin{enumerate}
\item If the file h\textsubscript{i} has a reference count greater than or equal to \textbf{SNAPPER\_REDUND}:
\begin{enumerate}
\item Create a new file h\textsubscript{j} as outlined in Step 6.
\item Increment the reference count for all files in \texttt{Snapper::Archiver[i]}.
\item Update \texttt{Snapper::Archiver[i]} \(\gets\) (\emph{path(} h\textsubscript{j} \emph{)} || ',' || \texttt{Snapper::Archiver[i]}) - i.e. prepend the new file path, separated by a comma, to the string containing the redundant file copies.
\end{enumerate}
\item If the file h\textsubscript{i} has a reference count lower than \textbf{SNAPPER\_REDUND}, increment the reference count of it and all other redundant files in \texttt{Snapper::Archiver[i]}.
\end{enumerate}
\item Save \texttt{Snapper::Archiver} into the archive file and calculate its CRC.
\item Rename "current" to the current UNIX timestamp to signify that the generation is complete.
\end{enumerate}
\section{{\bfseries\sffamily TODO} \framebox{\#B} Snapshot Recovery}
\label{sec:orgb74ba94}
\noindent\textbf{DEADLINE:} \textit{<2025-06-23 Mon>}\\
\begin{verbatim}
Effort: 10
\end{verbatim}
\begin{enumerate}
\item Choose a generation to boot from (by default the latest one).
\item Check if the generation is valid (i.e. has an archive file with a valid CRC). If not, recovery is not possible.
\item Load the archive file of the latest valid generation into \texttt{Snapper::Archiver}.
\item For each h \(\in\) \texttt{Snapper::Archiver} and for each redundant file, h\textsubscript{i} \(\in\) h:
\begin{enumerate}
\item Check the CRC with the stored data.
\item If h\textsubscript{i} does not exist or there is a mismatch with the CRC, try the next redundant file.
\item If there are no more redundant files to check, respond according to the configured policy.
\item If the CRC matches h\textsubscript{i}, load the data of h\textsubscript{i} into the corresponding page.
\end{enumerate}
\end{enumerate}
\section{{\bfseries\sffamily TODO} \framebox{\#C} Snapshot Purge}
\label{sec:orga64aeb8}
\noindent\textbf{DEADLINE:} \textit{<2025-06-23 Mon>}\\
\begin{verbatim}
Effort: 10
\end{verbatim}
Note, that when a file's reference count is decremented to 0, the file is removed. If a directory becomes empty as a result, it is removed.

\begin{enumerate}
\item Make sure the generation is valid (i.e. it has an archive file with a valid CRC).
\item If the archive file has an invalid CRC:
\begin{enumerate}
\item If \textbf{SNAPPER\_INTEGR} is set to true, crash the system and ask the system administrator to replace the generation's corrupted archive file with a backup copy.
\item Otherwise, log an error message and boot the system into a clean state.
\end{enumerate}
\item If the archive file has a valid CRC:
\begin{enumerate}
\item Load the archive file into \texttt{Snapper::Archiver}.
\item For each entry h \(\in\) \texttt{Snapper::Archiver} and for each file h\textsubscript{i} \(\in\) h: decrement the file h\textsubscript{i}'s reference count.
\item Delete the archive file.
\end{enumerate}
\end{enumerate}
\section{{\bfseries\sffamily TODO} \framebox{\#C} XML Configuration Support}
\label{sec:org0d4afce}
\noindent\textbf{DEADLINE:} \textit{<2025-06-30 Mon>}\\
\begin{verbatim}
Effort: 5
\end{verbatim}
\begin{itemize}
\item[{$\square$}] SNAPPER\_ROOT
\item[{$\square$}] SNAPPER\_THRESH
\item[{$\square$}] SNAPPER\_INEGR
\item[{$\square$}] SNAPPER\_REDUND
\item[{$\square$}] Retention::MAX\_SNAPS
\item[{$\square$}] Retention::EXPIRATION
\end{itemize}
\section{{\bfseries\sffamily TODO} \framebox{\#C} Integration Into PhantomOS}
\label{sec:org13194ca}
\noindent\textbf{DEADLINE:} \textit{<2025-06-30 Mon>}\\
\begin{verbatim}
Effort: 10
\end{verbatim}
\section{{\bfseries\sffamily TODO} \framebox{\#C} PhantomOS Snapshot Tests}
\label{sec:orgb836a2f}
\noindent\textbf{DEADLINE:} \textit{<2025-07-07 Mon>}\\
\begin{verbatim}
Effort: 5
\end{verbatim}
\begin{itemize}
\item[{$\square$}] Snapshot creation
\item[{$\square$}] Snapshot recovery
\item[{$\square$}] Snapshot purge
\end{itemize}
\end{document}
